\documentclass[../main.tex]{subfiles}
\graphicspath{ {../img/} }


\begin{document}

    \newpage
    \pagenumbering{arabic}

	\chapter*{Abstract}
	\addcontentsline{toc}{chapter}{Abstract}

    \section*{Abstract}

    %TODO
Whatever one host can do by itself, one thousand hosts can do it better together. Malicious actors spread their networks of zombie hosts to grant them the power of many machines.
They pose a challenge to law enforcement, companies and targeted individuals given their power to cause damage as well as resilience from being disrupted.

We decided to learn about the intricacies of this threat by creating our own botnet, written in Rust.
The aim was to delve into the low level world of programming networking functionality, hence why Rust was used. We used the Rust library documentation and Rust's compiler to learn
the language along the way and getting acquainted with how a networking program should be written in a low level.
This botnet utilizes the ShellShock ie. CVE-2014-6271 on hosts running an outdated version of Apache.
Communication is encrypted with TLS thanks to OpenSSL and Rustls. By design, only 1 to 1 parent-child node relations are possible, owing to the targeting system implemented.

The exploit itself was just as important as the botnet itself. Without this, the hosts could never have been infected by its parent server, andso would never have joined the network
of zombie machines.

Despite the age of the exploit, it is very likely that many machines and running this outdated version of Apache, meaning that our malware could potentialy ensnare multiple hosts,
adding them to our botnet. Therefore it becomes clear that botnets, even primitive ones by today's standards, pose a threat to security for anyone and everyone, and that botnet
detection algorithms must implemented and machines up to date to prevent infection/exploitation.



	\qquad \textit{Keywords}: Botnet, malware, virus, rust, TODO.


	\vspace{10pt}

\end{document}
