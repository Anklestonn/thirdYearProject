\documentclass[../main.tex]{subfiles}
\graphicspath{ {../img/} }


\begin{document}


	\chapter*{Conclusion}
	\addcontentsline{toc}{chapter}{Conclusion}


    %TODO
Our project has certainly identified and demystified the processes required for a botnet/worm combination to function, as well as showing the development of this architecture. 
Despite the fact that we had no experience in Rust, low level programming, malware development and , to some extent, network programming, we managed to complete our intended goal of developing the malware.
This clearly means that a botnet can be made by anyone with enough patience/perseverance to learn a programming language and fill in the gaps in their own skillset by consulting the documentatio of the language.
This, however, raises the question of how much more could we have achieved if we had the experience in network and malware programming?
If we were to re-attempt this project in a month's time then there would be a marked decrease in initial development time and more advanced features, such as decentralization could be achieved within the same time frame.
Although the botnet was a crucial aspect of the malware it wasnt what made the malware work, per se, though it was merely a chassis for the bot network to form. 
The engine or propagator was, in fact, Shellshock.
It was the development of the botnet that allowed to exploit to do its job, and without the botnet, the exploit wouldn't be nearly as dangerous.
Anyone can find on the internet, using github.com or vx-underground.org, a working exploit and malware to repurpose for their own code.

Possible ways to expand on this idea of building a botnet for educational means would be to make a botnet with more advanced features, or to build a botnet while writing minimal amounts of code, ie. using premade tools.
A project researching botnet development using only premade tools would be pertinent to CyberSecurity due to the threat of script kiddies, which are generally less skilled programmers, if they program at all.

Therefore, the process of developing a botnet has been made clear with our efforts; botnets are a persistent threat and are relatively easy to develop.

	\vspace{10pt}


\end{document}
