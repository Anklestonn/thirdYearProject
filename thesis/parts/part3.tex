\documentclass[../main.tex]{subfiles}
\graphicspath{ {../img/} }


\begin{document}


	\chapter{Our botnet limitations.}


    Our worm botnet was created specifically for demonstration purposes and for proof of concept. 
    Moreover, it's not intended for large-scale or sophisticated attacks. 
    Due to the botnet's limitations and design, it is not able to attack on these levels. 
    For example, the botnets targeting IP system is undesirable due to its inefficiency. 
    Its number of potential targets is limited due to having to choose the IP beforehand and only being able to attack the first IP within the list of IPs. 
    Furthermore, if for a demonstration this is enough, for a real-world botnet situation it could potentially be a critical design flaw because it won't check if the targeted host is vulnerable before it executes. 

    Within our time frame for this project, we were only able to design the botnet so that it is only capable of loading one exploit at a time. 
    Therefore, it greatly reduces versatility and scalability within our botnet. 
    Propagating is another aspect which shows that we could better our design, the time that the attack order takes to spread to the last node, is far too delayed. 
    This inevitably affects attacks across all nodes.

    Another limitation of our botnet is on the client side. 
    The client asks for order1 then order2, this is regardless of if order1 could even be downloaded and executed properly. 
    Thus, showing how this lack of error handling is insufficient causing problems within the attack process. 

    Our choice of ports 7878 and 7870 is poor because in real world situations our server should only have one port, making it more efficient and flexible. 
    The file sharing server and how it handles binary transfers is inefficient and should be merged to the CC server.
    Another limitation is that our botnet is only linux compatible
    The installation process is inefficient, all downloaded files should be automatically be made accessible for the next node, instead of relying on the installation, and attack script. 
    For encryption we made it with self-signed certificates, the more adequate design would be to make our own root certificate and assign them that way. 
    This way there is a hierarchical trust model therefore enhancing security.



\end{document}
