\documentclass[../main.tex]{subfiles}
\graphicspath{ {../img/} }


\begin{document}

	\chapter*{Refenrences}
	\addcontentsline{toc}{chapter}{References}
	{

        [1] wikipedia page about worm malware. \url{https://en.wikipedia.org/wiki/Computer_worm}

        [2] HowToGeek website speeking about worm.
        \url{https://www.howtogeek.com/415873/what-are-internet-worms-and-why-are-they-so-dangerous/}

        [3] malwarebytes's definition of botnet.
        \url{https://www.malwarebytes.com/botnet}

        [4] research paper on peer-to-peer botnets.
        \url{https://ieeexplore.ieee.org/abstract/document/5684002/}

        [5] study on worm propagation.
        \url{https://citeseerx.ist.psu.edu/document?repid=rep1&type=pdf&doi=5de6987574e48076fa5f024f347d44c77a6fa080}

        [6] wikipedia page about wanacry worm.
        \url{https://en.wikipedia.org/wiki/WannaCry_ransomware_attack}

        [7] csoonline article about wanacry.
        \url{https://www.csoonline.com/article/563017/wannacry-explained-a-perfect-ransomware-storm.html}

        [8] wikipedia page about mirai botnet.
        \url{https://en.wikipedia.org/wiki/Mirai_(malware)}

        [9] wikipedia page about hajime botnet.
        \url{https://en.wikipedia.org/wiki/hajime_(malware)}

        [10] thehackernews page speeking about the hajime botnet.
        \url{https://thehackernews.com/2017/04/vigilante-hacker-iot-botnet_26.html}

        [11] Opsxcq github repository about CVE-2014-6271 Shellshock.
        \url{https://github.com/opsxcq/exploit-CVE-2014-6271}

        [12] wireshark website link.
        \url{https://www.wireshark.org/}

        [13] dinit project website.
        \url{https://davmac.org/projects/dinit/}

		\vspace{10pt}

	}


\end{document}
